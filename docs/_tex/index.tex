% Options for packages loaded elsewhere
% Options for packages loaded elsewhere
\PassOptionsToPackage{unicode}{hyperref}
\PassOptionsToPackage{hyphens}{url}
\PassOptionsToPackage{dvipsnames,svgnames,x11names}{xcolor}
%
\documentclass[
  letterpaper,
  DIV=11,
  numbers=noendperiod]{scrartcl}
\usepackage{xcolor}
\usepackage{amsmath,amssymb}
\setcounter{secnumdepth}{-\maxdimen} % remove section numbering
\usepackage{iftex}
\ifPDFTeX
  \usepackage[T1]{fontenc}
  \usepackage[utf8]{inputenc}
  \usepackage{textcomp} % provide euro and other symbols
\else % if luatex or xetex
  \usepackage{unicode-math} % this also loads fontspec
  \defaultfontfeatures{Scale=MatchLowercase}
  \defaultfontfeatures[\rmfamily]{Ligatures=TeX,Scale=1}
\fi
\usepackage{lmodern}
\ifPDFTeX\else
  % xetex/luatex font selection
\fi
% Use upquote if available, for straight quotes in verbatim environments
\IfFileExists{upquote.sty}{\usepackage{upquote}}{}
\IfFileExists{microtype.sty}{% use microtype if available
  \usepackage[]{microtype}
  \UseMicrotypeSet[protrusion]{basicmath} % disable protrusion for tt fonts
}{}
\makeatletter
\@ifundefined{KOMAClassName}{% if non-KOMA class
  \IfFileExists{parskip.sty}{%
    \usepackage{parskip}
  }{% else
    \setlength{\parindent}{0pt}
    \setlength{\parskip}{6pt plus 2pt minus 1pt}}
}{% if KOMA class
  \KOMAoptions{parskip=half}}
\makeatother
% Make \paragraph and \subparagraph free-standing
\makeatletter
\ifx\paragraph\undefined\else
  \let\oldparagraph\paragraph
  \renewcommand{\paragraph}{
    \@ifstar
      \xxxParagraphStar
      \xxxParagraphNoStar
  }
  \newcommand{\xxxParagraphStar}[1]{\oldparagraph*{#1}\mbox{}}
  \newcommand{\xxxParagraphNoStar}[1]{\oldparagraph{#1}\mbox{}}
\fi
\ifx\subparagraph\undefined\else
  \let\oldsubparagraph\subparagraph
  \renewcommand{\subparagraph}{
    \@ifstar
      \xxxSubParagraphStar
      \xxxSubParagraphNoStar
  }
  \newcommand{\xxxSubParagraphStar}[1]{\oldsubparagraph*{#1}\mbox{}}
  \newcommand{\xxxSubParagraphNoStar}[1]{\oldsubparagraph{#1}\mbox{}}
\fi
\makeatother


\usepackage{longtable,booktabs,array}
\usepackage{calc} % for calculating minipage widths
% Correct order of tables after \paragraph or \subparagraph
\usepackage{etoolbox}
\makeatletter
\patchcmd\longtable{\par}{\if@noskipsec\mbox{}\fi\par}{}{}
\makeatother
% Allow footnotes in longtable head/foot
\IfFileExists{footnotehyper.sty}{\usepackage{footnotehyper}}{\usepackage{footnote}}
\makesavenoteenv{longtable}
\usepackage{graphicx}
\makeatletter
\newsavebox\pandoc@box
\newcommand*\pandocbounded[1]{% scales image to fit in text height/width
  \sbox\pandoc@box{#1}%
  \Gscale@div\@tempa{\textheight}{\dimexpr\ht\pandoc@box+\dp\pandoc@box\relax}%
  \Gscale@div\@tempb{\linewidth}{\wd\pandoc@box}%
  \ifdim\@tempb\p@<\@tempa\p@\let\@tempa\@tempb\fi% select the smaller of both
  \ifdim\@tempa\p@<\p@\scalebox{\@tempa}{\usebox\pandoc@box}%
  \else\usebox{\pandoc@box}%
  \fi%
}
% Set default figure placement to htbp
\def\fps@figure{htbp}
\makeatother





\setlength{\emergencystretch}{3em} % prevent overfull lines

\providecommand{\tightlist}{%
  \setlength{\itemsep}{0pt}\setlength{\parskip}{0pt}}



 


\KOMAoption{captions}{tableheading}
\makeatletter
\@ifpackageloaded{caption}{}{\usepackage{caption}}
\AtBeginDocument{%
\ifdefined\contentsname
  \renewcommand*\contentsname{Table of contents}
\else
  \newcommand\contentsname{Table of contents}
\fi
\ifdefined\listfigurename
  \renewcommand*\listfigurename{List of Figures}
\else
  \newcommand\listfigurename{List of Figures}
\fi
\ifdefined\listtablename
  \renewcommand*\listtablename{List of Tables}
\else
  \newcommand\listtablename{List of Tables}
\fi
\ifdefined\figurename
  \renewcommand*\figurename{Figure}
\else
  \newcommand\figurename{Figure}
\fi
\ifdefined\tablename
  \renewcommand*\tablename{Table}
\else
  \newcommand\tablename{Table}
\fi
}
\@ifpackageloaded{float}{}{\usepackage{float}}
\floatstyle{ruled}
\@ifundefined{c@chapter}{\newfloat{codelisting}{h}{lop}}{\newfloat{codelisting}{h}{lop}[chapter]}
\floatname{codelisting}{Listing}
\newcommand*\listoflistings{\listof{codelisting}{List of Listings}}
\makeatother
\makeatletter
\makeatother
\makeatletter
\@ifpackageloaded{caption}{}{\usepackage{caption}}
\@ifpackageloaded{subcaption}{}{\usepackage{subcaption}}
\makeatother
\usepackage{bookmark}
\IfFileExists{xurl.sty}{\usepackage{xurl}}{} % add URL line breaks if available
\urlstyle{same}
\hypersetup{
  pdftitle={Pregnancy Complicated by Klippel--Trenaunay Syndrome and Kasabach Merritt syndrome Resulting in Severe Fetal Growth Restriction: A case report and review of literature},
  colorlinks=true,
  linkcolor={blue},
  filecolor={Maroon},
  citecolor={Blue},
  urlcolor={Blue},
  pdfcreator={LaTeX via pandoc}}


\title{Pregnancy Complicated by Klippel--Trenaunay Syndrome and Kasabach
Merritt syndrome Resulting in Severe Fetal Growth Restriction: A case
report and review of literature}
\author{}
\date{}
\begin{document}
\maketitle


\section{Authors}\label{authors}

Mourad Elfaham\textsuperscript{1}, Aya Attia\textsuperscript{1}, Sara
Ibrahim Abdelkader\textsuperscript{2*}, Rahma Alaa
Abdelhafez\textsuperscript{2}, Mohammed Ghanem\textsuperscript{2}, Ahmed
Azab\textsuperscript{2}, Belgin Ahmed Nagah\textsuperscript{2}, Nour
Atif\textsuperscript{2}, Maha Moemen\textsuperscript{2}, Ashraf
Nabhan\textsuperscript{1,3}

\section{Affiliation}\label{affiliation}

\textsuperscript{1} Faculty of Medicine, Ain Shams University, Cairo,
Egypt.

\textsuperscript{2} Faculty of Medicine, Galala University, Attaka,
Suez, Egypt.

\textsuperscript{3} Faculty of Medicine, MTI University, Cairo, Egypt.

* Corresponding author: Sara Ibrahim Abdelkader. Faculty of Medicine,
Galala University, Attaka, Suez, Egypt. Email:
\href{mailto:sara.elzeftawy@gu.edu.eg}{\nolinkurl{sara.elzeftawy@gu.edu.eg}},
ORCID: \href{https://orcid.org/0009-0002-5877-4722}{0009-0002-5877-4722}

Word count: 2500

\section{Abstract}\label{abstract}

Klippel--Trenaunay syndrome (KTS) is a rare congenital vascular disorder
that may be complicated by Kasabach Merritt syndrome, posing significant
maternal and fetal risks during pregnancy.

We report the case of a 28-year-old pregnant woman with known
Klippel--Trenaunay syndrome who developed Kasabach--Merritt phenomenon
during pregnancy. The pregnancy was complicated by severe fetal growth
restriction (FGR) and absent end-diastolic flow on umbilical artery
Doppler. A cesarean section was performed at 35 weeks of gestation. A
growth-restricted neonate was delivered and managed accordingly.

This case highlights the complexity of managing pregnancy in patients
with KTS complicated by KMP and emphasizes the importance of
multidisciplinary surveillance and timely delivery.

Keywords: Klippel--Trenaunay syndrome; angioosteohypertrophy syndrome;
Kasabach Merritt syndrome; fetal growth restriction; high-risk pregnancy

\section{Introduction}\label{introduction}

Klippel-Trenaunay-Weber syndrome is a sporadic genetic syndrome
characterized by localized hemangiomas, venous varicosities, and
asymmetric osseous hypertrophy of the ipsilateral extremities. Most
commonly seen in association with hemangiomas, Kasabach-Merritt syndrome
is defined by the presence of thrombocytopenia and a consumptive
coagulopathy. TK syndrome is a sporadic genetic syndrome characterized
by localized cutaneous hemangiomas, venous varicosities, and asymmetric
osseous hypertrophy of the ipsilateral extremities.1 Complications at
delivery can include disseminated intravascular coagulation (DIC) and
massive hemorrhage. Coagulopathic events may be due to persistent blood
loss or may reflect the development of Kasabach-Merritt syndrome. First
reported in 1940, this syndrome has been described most commonly in
children; it is defined by thrombocytopenia of varying degrees and a
consumptive coagulopathy seen in association with hemangiomas.2 A
microangiopathic hemolytic anemia may also be present.

Fetal growth restriction (FGR) with abnormal umbilical artery Doppler
findings, such as absent end-diastolic flow, reflects placental
insufficiency and is associated with adverse perinatal outcomes. We
report a rare case of pregnancy complicated by KTS and KMP resulting in
severe FGR requiring preterm cesarean delivery.

To our knowledge, only one case of a pregnancy complicated by
Klippel-Trenaunay-Weber and subsequent Kasabach-Merritt syndrome has
been reported.

\section{Case Presentation}\label{case-presentation}

A 28-year-old multigravida woman (G3, P2) presented at 35 weeks'
gestation for antenatal evaluation. She had a known diagnosis of
Klippel--Trenaunay syndrome involving {[}affected limb/region{]},
diagnosed at {[}age/year{]}.

Maternal Clinical Findings

• Blood pressure: {[}value{]}

• Presence of vascular malformations: {[}description{]}

• Symptoms suggestive of KMP: {[}e.g., bruising, bleeding, pain{]}

Laboratory investigations revealed:

• Platelet count: {[}value{]}

• Fibrinogen: {[}value{]}

• D-dimer: {[}value{]}

These findings were consistent with Kasabach--Merritt phenomenon.

Fetal Assessment

Ultrasound examination at {[}gestational age{]} weeks demonstrated:

• Estimated fetal weight below the {[}percentile{]}

• Abnormal umbilical artery Doppler with absent end-diastolic blood flow
• Amniotic fluid index: {[}value{]}

Based on worsening fetal Doppler parameters and maternal risks, a
multidisciplinary team involving obstetrics, hematology, anesthesia, and
neonatology recommended delivery.

Delivery and Neonatal Outcome

A cesarean section was performed at 35 weeks of gestation under {[}type
of anesthesia{]}. A female/male neonate weighing {[}birth weight{]} g
was delivered with Apgar scores of {[}values{]} at 1 and 5 minutes.

The neonate required {[}NICU admission / respiratory support /
observation{]}. Maternal postoperative recovery was
{[}uneventful/complicated by\ldots{]}.

Maternal admission to ICU

{[}course in details{]}

\section{Discussion}\label{discussion}

Klippel-Trenaunay syndrome, first described in 1900, is defined as a
triad of clinical features, including unilateral cutaneous hemangiomas,
varicose veins, soft tissue, and asymmetric osseous hypertrophy of the
ipsilateral extremities.! Weber'3 described several similar cases and,
in 1918, reported the additional feature of arteriovenous fistulae. A
review· of 768 cases of Klippel-Trenaunay-Weber syndrome described a
consistent absence or major atresia of the deep venous vascular system,
causing limb elongation, venous stasis, varicosities, and edema. A
venogram had previously demonstrated a similar absence of the deep
venous vascular system of the right lower extremity in our patient. The
etiology of Klippel-Trenaunay-Weber syndrome has been reviewed by
Baskerville et al,5 who suggested that a mesodermal defect operative at
angiogenesis may explain the vascular abnormalities that define the
syndrome. Complications in the affected extremity may include stasis
dermatitis, skin ulceration, and recurrent hemarthrosis-all of which may
require amputation of the limb for management. Our patient's refractory
coagulopathy prompted the diagnostic consideration of Kasabach-Merritt
syndrome. First reported in 1940, the syndrome consists of
thrombocytopenia and a consumptive coagulopathy seen in association with
hemangiomas? Consistent laboratory abnormalities include
thrombocytopenia, decreased fibrinogen, and coagulation factors II, V,
and Vill. Increased PT, PIT, and fibrin split products are also
invariably present. A similar clinical presentation has been reported in
association with Klippel-Trenaunay-Weber syndrome.6 The exact mechanism
by which the thrombocytopenia and coagulopathy develop is unknown. It
has been suggested that a combination of venous stasis and abnormal
endothelial cells within the hemangiomas functions to increase platelet
pooling and destruction with activation of the coagulation cascade,
resulting in a localized and self-perpetuating coagulopathy.7 Inceman
and Tangun8 reported a case in which an 8-yearold boy with a giant
hemangioma of the lower extremity demonstrated a difference in PT and
PIT values in blood obtained from a peripheral site in comparison with
blood obtained from the hemangioma. We suspect that a similar mechanism
occurred in our patient. Despite choosing a left paramedian incision, a
subtle disruption of the right-sided hemangioma, exacerbated by
increased ambulation, may have initiated the coagulopathy. Once
established, the local consumption of clotting factors and platelets
persisted in the hemangioma and resulted in hemorrhage at the incision
site. A number of therapeutic modalities have been used in the
management of Kasabach-Merritt syndrome. These include platelet
transfusion, IV heparin, aminocaproic acid, radiotherapy, alpha 2a
interferon, corticosteroids, and intermittent pneumatic compression of
the extremities. A prominent feature of this patient's persistent
coagulopathy was the markedly elevated fibrin split products. This
finding prompted our decision to use the plasminogen-activator
inhibitory effects of aminocaproic acid after the failure of blood
product replacement therapy to correct her coagulopathy. Although a
pregnancy presenting with Klippel-Trenaunay-Weber syndrome is rare, many
patients with known deep or superficial hemangiomas can and do become
pregnant. The potential for a refractory coagulopathy presenting as
Kasabach-Merritt syndrome should be considered in any patient who
presents with extensive hemangiomas.

\section{Conclusion}\label{conclusion}

Pregnancy in women with Klippel--Trenaunay syndrome complicated by
Kasabach--Merritt phenomenon is rare and high risk. Intensive antenatal
surveillance and multidisciplinary management are essential to optimize
maternal and perinatal outcomes.

\section{Declarations}\label{declarations}

\subsection{Ethics approval and consent to
participate}\label{ethics-approval-and-consent-to-participate}

Not applicable.

\subsection{Consent for publication}\label{consent-for-publication}

A written consent for publication has been obtained from the patient.

\subsection{Availability of data and
materials}\label{availability-of-data-and-materials}

The data and materials supporting the conclusions of this article are
available as a supplementary materials.

\subsection{Competing interests}\label{competing-interests}

The authors declare that they have no competing interests.

\subsection{Funding}\label{funding}

This research received no specific grant from any funding agency in the
public, commercial or not-for-profit sectors.

\subsection{Authors' contributions}\label{authors-contributions}

SIA, ME, AFN contributed to writing the first draft of the manuscript.
All the authors revised the manuscript critically for important
intellectual content. All authors approved the final version of the
manuscript.

\subsection{Acknowledgements}\label{acknowledgements}

We sincerely thank the healthcare providers at Ain Shams University
Hospital of Obstetrics and Gynecology for their dedication and
commitment to patient care in the operating room. We deeply appreciate
their contributions and the valuable role they play in ensuring
high-quality surgical care.

\section{References}\label{references}




\end{document}
